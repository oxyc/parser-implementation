\begin{abstract}{swedish}
  Avsikten med detta examensarbete är att redogöra för hur parsningen av ett
  programmeringsspråk fungerar samt att undersöka varför vissa kompilatorer
  implementerar en handskriven parser medan andra använder en maskingenererad
  parser.
  Examensarbetet består av en teoretisk grund samt en praktisk implementation av
  en handskriven Lua-parser. Den teoretiska delen beskriver hur
  programmeringsspråk är uppbyggda samt ger en överblick av hur en parser
  implementeras. Teknikerna som tas upp fokuserar huvudsakligen på handskrivna
  parsers. Den praktiska implementationen är programmerad i JavaScript för att
  kunna användas som ett analyseringsverktyg i en nätbaserad
  programkodsredigerare.  Problemen som påträffas under implementationen
  behandlas med hjälp av tekniker presenterade i teorin.  Slutligen genomgår
  parser-implementationen en prestandaoptimering för JavaScript-motorn V8.
  Resultatet jämförs med en maskingenererad parser samt med en handskriven
  parser implementerad i Lua.  Slutsatsen är att handskrivna parsers kan uppnå
  högre prestanda samt ökad flexibilitet men kräver avsevärt mera tid att
  implementera än en maskingenererad parser.
  \\
  \\
\end{abstract}

\begin{abstract}{english}
  The purpose of this thesis is to describe the process of parsing a
  programming language and investigate why some compilers implement their
  own handwritten parser while others use a machine generated parser.
  The thesis consists of a theoretical foundation and a practical
  implementation of a handwritten Lua parser.  The theoretical part gives an
  overview of how programming languages are built and how a parser is
  implemented.
  The techniques presented focus primarily on handwritten parsers.
  The practical parser implementation is written in JavaScript so that it can
  be used as an analysis tool within online code editors.
  The problems encountered during the implementation process are solved with
  the help of techniques presented in the theoretical part. At the end the
  parser undergoes a performance optimization for the V8 JavaScript engine.
  The achieved results are compared to a machine generated parser as well as
  to a handwritten parser implemented in Lua. The conclusion is that a
  handwritten parser can achieve higher performance and an increased
  flexibility but is significantly more time consuming to implement than a
  machine generated parser.
  \\
  \\
\end{abstract}

% vim: set tw=78:ts=2:sw=2:et:fdm=marker:wrap:wm=78:ft=tex
% vim: spell spelllang=sv
