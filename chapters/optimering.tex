\section{Prestandaoptimering}

Eftersom JavaScript är ett tolkat språk som evaluerar kod vid exekvering bör
tolkade språk fokusera mer tid på prestandaoptimering än kompilerade språk. Dessutom är
JavaScript ett språk som huvudsakligen körs i webbläsare med varierande
prestanda. Mindre applikationer med ett fåtal funktionsanrop behöver vanligen
inte prioritera prestandaoptimering eftersom webbläsarens JavaScript-motor är
tillräkligt effektiv för att inte påverka användarupplevelsen. Målet med detta
examensarbete handlar dock om att implementera en parser som kan användas i
nätbaserade kodredigerare, exempelvis som en syntaxmarkerare. Eftersom en
syntaxmarkerare behöver uppdateras vid varje kodförändring kan antalet
funktionsanrop snabbt stiga och påverka användarupplevelsen. När användaren
gör en uppdatering förväntas en omedelbar respons och därmed måste en parser
ämnad för webbläsare uppfylla detta krav.

Insamligen av data utgörs av att parsa ett exempelprogram vid specifika
revisioner av implementation. Utgående från resultatet uppdateras
implementationen och det nya resultatet granskas.

\subsection{JavaScript-motorn V8}

För att möjliggöra automation av provtaging utförs dessa test med Node.js
som använder sig av JavaScript-motorn V8. V8 är en JavaScript-motor med öppen
källkod som skapats av Google ämnad för deras webbläsare Google Chrome.

V8 består av en optimeringskompilator kallad Crankshft. Syftet med denna
kompontent är att identifiera och optimera enbart de funktioner som är
väsentliga för exekveringen av JavaScript-kod. Processen att optimera kod i en
tolk är tidskrävande och därmed bör optimeringen enbart fokusera på funktioner
som har en inverkan på exekveringstiden.

Processen för Crankshaft utgörs av fyra steg där det första innebär att
baskompilatorn kompilerar koden som normalt. Efter detta börjar en s.k.
\textit{``runtime-profiler''} övervaka körningen av programmet för att
identifiera de väsentliga funktionerna. När en funktion anses vara väsentlig
för exekveringen skickas börjar en s.k. girig omkompilering.

En girig kompilering innebär att kompilatorn inte bryr sig om ifall
optimeringen kommer att lyckas eller inte. Istället för att analysera koden
väljer kompilatorn att testa sig fram och sedan återgå till den ursprungliga
koden om optimeringen misslyckades \citep{mk10}. Denna åtgärd kallas för
avoptimering och inträffar aktivt i körningen av JavaScript eftersom språket
är dynamiskt och en avoptimering sker varje gång en datatyp konverteras.
Ytterligare existerar det specifika strukturer som kompilatorn inte kan
optimera, exempelvis try-satser samt vissa typer av switch-satser. Även här
måste en avoptimering ske men för dessa struktur-relaterade problem kallas det
för en s.k. \textit{``bailout''}.

För att identifiera dessa avoptimeringar och \textit{``bailouts''} har V8 diverse
kommandoradsflaggor som kan användas. Dessa kan även användas för att
inspektera funktionsanrop och händelseförlopp.

\subsubsection{Kommandoradsflaggor}

Nedan följer en lista på de kommandoradsflaggor som används i detta kapitel
\citep{v8profiler}.

\begin{description}
  \item[--prof] \hfill \\
    Skapar en v8.log fil bestående av provtagning från JavaScript-stacken
    samt C++-stacken. Med hjälp av verktyget tick-processor som följer med V8
    kan användaren skapa en översikt av denna information. Översikten återger
    hur stor andel av den totala exekveringen som utgörs av varje funktion.
  \item[--trace_bailout] \hfill \\
    Anger orsaken till varför en funktion avoptimerats med en
    \textit{``bailout''}.
  \item[--trace_inlining] \hfill \\
    Registrerar varje gång en Crankshaft försöker omkompilera en funktion med
    hjälp av en teknik som kallas \textit{``function inlining''}. Denna
    optimeringsteknik innebär att kompilatorn ersätter ett funktionsanrop med
    själva funktionsinnehållet och därmed sparar kostnaden av själva andropet
    samt dess returnering. När en \textit{``function inlining''} utförs
    möjliggör det ytterligare optimeringsmöjligheter eftersom koden nu är mer
    kompakt och introducerar satser som möjligen kan kombineras med omliggande
    kod.
\end{description}

\subsection{Provtagning}

Följande moment har genomförts för att optimera prestandan av parsern fram
till att resultatet har en relativ förbättring större än 10\%. Alla test har
utförts på en dual-core 2.6 GHz dator med ett Linux-baserat operativsystem.

\subsubsection{Undvikandet av ``bailouts''}

Trots att \textit{``bailouts''} möjliggör en allmän prestandaförbättring hos
V8 är det klokt att undvika avoptimeringen eftersom den kostar en del
prestanda och kommer att återupprepas vid varje exekvering.

Genom att analysera resultatet från en parsning med kommandoradsflaggan
\textit{--trace_bailout} i bilaga 6 kan två
\textit{``bailout''}-avoptimeringar utläsas.  Funktionerna som orsakat dessa
är \textit{isUnary()} samt \textit{parsePrimaryExpression()} som bägge
innehåller en ooptimerbar switch-sats. \mbox{Node.js} v0.8.16 har inte
möjlighet att optimera funktioner som innehåller en switch-sats med
icke-litterala labels, d.v.s. variabler.

Detta är ett enkelt problem att lösa eftersom avoptimeringen inte sker för
``if-sats''-motsvarigheten och därmed kan koden optimeras enligt
\nobreak{figur~\ref{fig:bailout}}.

\lstloadlanguages{JavaScript}
\begin{figure}[ht]
  \begin{minipage}[t]{0.5\textwidth}
    \lstinputlisting[title="",language=Javascript]%
      {figures/tex/bailout-pre.js}
  \end{minipage}%
  \begin{minipage}[t]{0.5\textwidth}
    \lstinputlisting[title="",language=Javascript]%
      {figures/tex/bailout-post.js}
  \end{minipage}
  \caption{Förändring av funktionen isUnary från icke-optimerbar (vänster) till
    optimerbar (höger).}
  \label{fig:bailout}
\end{figure}

\subsubsection{Teckenkoder istället för reguljära uttryck}

Genom att analysera resultat i bilaga 7 som fås från provtagningsverktyget i
V8 framgår det att 17.6\% (@TODO data uppdateras ännu) av en hel exekvering
går åt till reguljära uttryck som används för att identifiera blanksteg och
giltiga identifierare.

Det första som kan göras är att omvandla dessa reguljära uttryck till vanliga
funktioner som granskar tecken. I provtagningsresultatet framgår det också att
\textit{readToken()} som är lexer-funktionen ansvarig för att identifiera
tokens upptar 30\% (@TODO data uppdateras ännu)  av hela parsningsprocessen.
Eftersom detta är en väsentlig del av exekveringstiden är det värt att
optimera alla teckenjämförelser till att istället arbeta med teckenkoder.
Detta gör att JavaScript-motorn kan använda sig av binära operationer för att
utföra en jämförelse. Denna metod visar sig ha en betydande prestandaskillnad
i äldre versioner av V8 samt i övriga webbläsare \citep{charcodeat}.

\subsubsection{Optimering av sträng-funktioner}

Efter att de tidigare optimeringarna genomförts betonas nu 2 huvudsakliga
funktioner i provtagningsanalysen i bilaga 8. Dessa funktioner, dvs.
\textit{StringAddStub} som är en maskinkodsmetod för att sammanfoga strängar
och \textit{CompareICStub} som är en maskinkodsmetod för att jämföra strängar,
upptar 8.3\% respektive 7.9\% av parsningsprocessens exekveringstid.

\textit{StringAddStub} orsakas av att lexern sammanfogar token-värden för
kommentarer, tal och identifierare genom att iterera över varje tecken så
länge som en avgränsare inte påträffas. Denna procedur går att förenkla genom
att istället memorera var värdet börjar och sedan iterera fram till
avgränsaren. När avgränsaren hittats kan värdet identifieras från dess
startposition till dess avgränsare enligt exemplet i figur~\ref{fig:stringAddStub}.

\lstloadlanguages{JavaScript}
\begin{figure}[ht]
  \begin{minipage}[t]{0.5\textwidth}
    \lstinputlisting[title="",language=Javascript]%
      {figures/tex/stringaddstub-pre.js}
  \end{minipage}%
  \begin{minipage}[t]{0.5\textwidth}
    \lstinputlisting[title="",language=Javascript]%
      {figures/tex/stringaddstub-post.js}
  \end{minipage}
  \caption{Optimering av vänstra funktionens strängsammanfogning till högra
    funktionens identifiering.}
  \label{fig:stringAddStub}
\end{figure}

Orsaken till den långsamma \textit{CompareICStub}-operationen visar sig vara
en \textit{``function inlining''} i uttrycksparsern. Funktionen ansvarar för
att ange prioriteten av en given operator utgående från en switch-sats
med strängjämförelser. Eftersom funktionen körs för varje uttryck som hittas
valde jag att uppoffra läsligheten med att istället jämföra teckenkoder en
efter en enligt figur~\ref{fig:switchtree}. Efter att optimeringen
gjorts valideras ännu att funktionen är tillräkligt liten för
\textit{``function inlining''}, vilket den visar sig vara.

\begin{figure}[ht]
  \begin{minipage}[t]{0.5\textwidth}
    \lstinputlisting[title="",language=Javascript]%
      {figures/tex/prefixtree-pre.js}
  \end{minipage}%
  \begin{minipage}[t]{0.5\textwidth}
    \lstinputlisting[title="",language=Javascript]%
      {figures/tex/prefixtree-post.js}
  \end{minipage}
  \caption{Optimering från jämförelse av tecken (vänster) till jämförelse av teckenkoder (höger).}
  \label{fig:switchtree}
\end{figure}

\subsection{Resultat}

(@TODO data uppdateras ännu)

\begin{figure}[ht]
  \includegraphics[width=13cm]{figures/output/functions.pdf}
  \caption{Slutgiltig översikt av exekveringstiden för parser-implementations
    funktioner.}
\end{figure}

\begin{figure}[ht]
  \includegraphics[width=15cm]{figures/output/commits.pdf}
  \caption{Översikt av moment i.o.m. parser-implementationens
    prestandaoptimering.}
\end{figure}

% vim: set tw=78:ts=2:sw=2:et:fdm=marker:wrap:wm=78:ft=tex
% vim: spell spelllang=sv
