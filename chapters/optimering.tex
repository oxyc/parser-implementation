\section{Prestandaoptimering}

Beskriv varför JavaScript behöver micro optimering.

\subsection{Insamling av data}

Beskriv processen för test samt sample standard deviation.

\subsection{Analysering av data för V8}

Beskriv varför v8.

\subsubsection{Sampling}

Beskriv samplings verktyget och hur datat läses.

\begin{figure}[ht]
  \includegraphics[width=13cm]{figures/output/functions.pdf}
  \caption{Översikt av de mest prestanda krävande funktionera}
\end{figure}

\subsection{Optimerings tekniker för V8}

Javascript motorn V8 innehåller en optimerings kompilator kallad Crankshaft,
vars syfte är att identifiera kod som körs mycket för att sedan spendera sin
tid på att optimera dom istället för de mindre väsentliga funktionerna. Detta
fungerar genom att baskompilern börjar med att kompilera koden varefter en
runtime profiler övervakar körningen av programmet och identifierar den kod
som körs aktivt. När de väsentliga programdelarna är identifierade
optimeras och omkompileras de ivrigt av optimerings kompilatorn. Innebördet
med att de omkompileras ivrigt är att det ytterligare finns en komponent som
kan återgå till baskompilatorns version om det visar sig att koden slutligen
inte var optimerbar \citep{mk10}. Denna åtgärd kallas för deoptimization och
inträffar aktivt i körningen av javascript eftersom språket är dynamiskt och
en deoptimization sker varje gång en datatyp förändras. Ytterligare existerar
det vissa specifika strukturer som kompilatorn inte kan optimera, exempelvis
try/catch-satser och vissa typer av switch-satser. Även här görs en
deoptimization men den kallas nu för en bailout. (@TODO terminology?)

V8 och därmed även Node.JS har diverse kommandorads flaggor som hjälper
identifiera funktioner var detta sker samt ge information om varför det sker.

\subsubsection{Undvikandet av bailouts}

Trots att bailouts möjliggör en allmän prestanda förbättring hos V8 är det
vettigt att undvika de bailouts som görs eftersom de vid varje exekvering
kommer återupprepa sig och processen med att återgå till den ursprungliga
kompilerings versionen kostar en del prestanda.

Kommandorads flaggan \textit{--trace_bailout} matar ut en lista på funktioner
samt orsaken till varför en bailout gjorts. Genom att exekvera en parsning med
flaggan aktiverad går det att identifiera två funktioner som gör en bailout,
bägge på grund av ooptimerbara switch statements var en icke-literal label
användts.

Funktionerna som orsakat dessa bailouts är \textit{isUnary()} samt
\textit{parsePrimaryExpression()} som bägge består av en switch-sats som utför
olika operationer beroende på om en given token är lika med ett visst
variabelvärde.  Detta är ett enkelt problem att lösa eftersom bailouten inte
sker för if-sats motsvarigheten och därför kan koden optimeras enligt
\nobreak{figur~\ref{fig:bailout}}.

\lstloadlanguages{JavaScript}
\begin{figure}[ht]
  \begin{minipage}[t]{0.5\textwidth}
    \lstinputlisting[title=a) Funktion före,language=Javascript]%
      {figures/tex/bailout-pre.js}
  \end{minipage}%
  \begin{minipage}[t]{0.5\textwidth}
    \lstinputlisting[title=b) Funktion efter,language=Javascript]%
      {figures/tex/bailout-post.js}
  \end{minipage}
  \caption{Förändring av funktionen isUnary från icke-optimerbar till
  optimerbar.}
  \label{fig:bailout}
\end{figure}

\subsubsection{Inlining}

@TODO expandera kanske? Hur översätta literaler?

Crankshaft använder sig även av optimeringstekniken funktion inlining. Detta
innebär att kompilatorn ersätter ett funktionsanrop med själva
funktionsinnehållet och sparar därmed kostnaden av själva funktions anropet
och dess returnering. Dessutom möjliggör det ytterligare optimeringstekniker
eftersom koden nu är mer kompakt och introducerar satser som möjligen kan
kombineras med omliggande kod.

V8 har dock vissa begränsningar för vilka funktioner som går att inlinea.
Bland annat får källkoden inte överstiga 600 bytes (hur skall jag referera
till v8 koden?) och funktioner kan inte heller innehålla array literaler. I V8
motorn som kommer med Node.JS v0.8 går det inte heller att inlinea funktioner
som innehåller objekt literaler.

Lexer implementationen som gjorts använde till att börja med två separata
funktioner som skapade och avslutade token noder. Eftersom noderna var
uppgjorda av objekt kunde dessa inte inlineas och därmed introducerades en
funktionskallelse extra för varje token som hittades under en parsning. På
grund av jag inte ansåg detta påverka läsligheten eller komplexiteten av koden
ansåg jag det vara värt att expandera skapandet och avslutandet av noderna i
varje enskild token funktion.

\subsubsection{Teckenkoder istället för reguljära uttryck}

Med användning av samplings verktyget i V8 är det möjligt att få en överblick
av vilka funktioner och operationer som är mest aktiva under en exekvering.
Genom att analysera detta resultat uppgavs det att 39.2\% av en hel exekvering
går åt till reguljära uttryck som används för att identifiera whitespace och
giltiga identifierare.

Eftersom dessa operationer ansvarade för en extremt stor del av
parsningsprocessen omvandlades de till binära operationer som jämför
teckenkoder istället. Orsaken till att teckenkoder används istället för string
literals är att en kompilator måste omvandla tecken till siffror för att göra
jämförelser, denna operation visar sig ha en betydande prestanda skillnad i
äldre V8 versioner samt övriga webbläsare
(hur ska dessa refereras? http://jsperf.com/charcodeat-vs-string-comparison/2).

\subsubsection{Optimering av sträng funktioner}

Efter att de tidigare optimeringarna genomförts sticker nu 2 huvudsakliga
funktioner ut i samplings analysen. StringAddStub som är en maskinkods metod
för att sammanfoga strängar och CompareICStub som i en metod för att jämföra
strängar tar plats 3 och 4 med 8.3\% respektive 7.9\% av parsningsprocessen.

StringAddStub orsakas av att lexern sammanfoga token värden för kommentarer,
nummer och identifierare genom att iterera över varje tecken så länge som en
avgränsare inte hittas. Denna procedur går att försimpla genom att istället
räkna var värdet börjar och sedan iterera fram till avgränsaren och vid det
skedet skära ut värdet från dess start position till positionen var
avgränsaren enligt exempel figur~\ref{fig:stringAddStub}.

\lstloadlanguages{JavaScript}
\begin{figure}[ht]
  \begin{minipage}[t]{0.5\textwidth}
    \lstinputlisting[title=a) Procedur före,language=Javascript]%
      {figures/tex/stringaddstub-pre.js}
  \end{minipage}%
  \begin{minipage}[t]{0.5\textwidth}
    \lstinputlisting[title=b) Procedur efter,language=Javascript]%
      {figures/tex/stringaddstub-post.js}
  \end{minipage}
  \caption{Procedur förändring från sträng sammanfogning till sträng
  utskärning.}
  \label{fig:stringAddStub}
\end{figure}

Vid vidare analys av sampling informationen gick det att utläsa att den
huvudsakliga orsaken för den långsamma CompareICStub operation var en funktion
som inlineats i uttrycks parsern. Funktionen ansvarar för att ange prioriteten
av en given binära operator utgående från en switch-sats med sträng
jämförelser. Eftersom funktionen körs för varje uttryck som hittas valde jag
att speciellt uppoffra läsligheten med att istället jämföra teckenkoder i
formen av ett träd enligt figur~\ref{fig:switchtree}. Efter att
optimeringen gjorts validerade jag ännu funktionen var tillräkligt liten för
att bli inlined, vilket den visade sig vara.

\begin{figure}[ht]
  \lstinputlisting[title="",language=Javascript]%
    {figures/tex/prefixtree.js}
  \caption{Optimerad träd implementation av en vanlig switch-sats för
    strängar.} \label{fig:switchtree}
\end{figure}

\subsubsection{Garbage Collection}

V8 består av två typer av garbage collectors som den väljer mellan att använda
under en exekvering. Den första är en scavanger som arbetar snabbt men enbart
på indirekt skapade objekt, dvs. objekt utan nyckelordet new. Den andra typen
av collector använder sig av en mark-sweep algoritm som även granskar indirekt
skapade objekt men är betydligt långsammare\citep{ma10}.

Eftersom parsern implementerats med hjälp av scopes och funktioner istället
för med ett objektorienterat paradigm är all allokering indirekt och därmed
finns det ingen oro om att minne skall läcka så länge som mer avancerade
closure användningar inte används. När en funktion körts färdigt blir alla
variabler som definierats inne i funktionen utom räckhåll för programmet och
därmed kommer de tas bort vid nästa scavanger runda. På grund av att minnet
knyts med hjälp av scopes är det därför viktigt att undvika globala variabler
samt se till att hela programmet är knytet till ett eget scope som utomstående
objekt inte har tillgång till.

Med hjälp av kommandorads flagga \textit{--trace_gc} kan vi se vilken typ av
garbage collection som görs samt hur länge den tagit.

Utgående från resultatet ser vi att det sker 215 scavange collections som alla
tar under 1ms att utföra, samt en mark-sweep som utförs tidigt i exekveringen
och kräver 7ms tid. Utgående från detta kan det konstateras att
implementationen inte har läcker minne och att scavange collectorn kan
upprätthålla programmet utan mer än en inledande mark-sweep.

\subsection{Resultat}

Beskriv vart vi kommit och varför vi inte kan optimera mer.

\begin{figure}[ht]
  \includegraphics[width=15cm]{figures/output/commits.pdf}
  \caption{Översikt av de mest prestanda påverkande kod commitsen}
\end{figure}

% vim: set tw=78:ts=2:sw=2:et:fdm=marker:wrap:wm=78:ft=tex
% vim: spell spelllang=sv
