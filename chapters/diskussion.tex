\section{Diskussion och slutsatser}

Slutresultatet av detta examensarbete är en prestandaoptimerad rekursivt
nedstigande parser av Lua-språket. Implementationen av parsern är en pågående
process som påbörjats 6 månader innan detta examensarbete fullgjorts.

Tidsmängden krävd för implementationen är omöjlig att mäta, men resultatet
utgörs av mer än 1600 källkodsrader samt mer än 600 logiska programkodsrader.
Ytterligare existerar det mer än 500 test som används för att upprätthålla
parserns kvalitet. I efterhand kan jag konstatera att implementationen varit
en okomplicerad men tidskrävande process. Eftersom tidsmängden för att
generera en Lua-parser med hjälp av Jison enbart tog enstaka timmar är
skillnaden mellan implementationsalternativen enorm. Slutprodukten visar sig
dock vara en stabil och prestandaeffektiv implementation som kan användas i
praktiken. Likt skaparna av GCC och Lua anser jag inte tidsmängden vara för
mycket när det handlar om ett språk med en stabil och enkel syntax. Fördelen
som fås visas tydligt i dess prestanda, dess felmeddelanden och enkelheten av
källkoden.

För språk med en mer komplicerad syntax såsom C++ eller PHP anser jag dock
inte det vara möjligt att upprätthålla en handskriven parser. Dessutom sker
det kontinuerligt uppdateringar i syntaxen av ett språk, vilket vi kan se i
senaste versionen av PHP samt i den uppkommande versionen av JavaScript.
Processen att uppdatera syntaxgrammatiken för en genererad parser innebär
enbart justeringar i en grammatikspecifikation. I en handskriven parser kan
det innebära stora strukturella förändringar.

Slutsatsen jag har är att projekt som körs i en prestandakritisk miljö med ett
tolkat språk bör överväga att implementera en handskriven parser. Om
tidsmängden är begränsad eller projektet inte är prestandakritiskt kan dock en
maskingenererad parser användas.

Det finns fortfarande en serie förbättringar som kan göras till denna
parser för att vara mer användbar som ett analyseringsverktyg. I nästa version
har jag planer att introducera inkrementell parsning så att syntaxmarkerare
inte behöver parsa en hel fil för att göra uppdateringar. Ytterligare planerar
jag anknyta källdkodspositionen till varje nod i syntaxträdet för att ett
verktyg lättare skall kunna manipulera programkod.

Slutprodukten av arbetet finns i \url{http://oxyc.github.io/luaparse/} och är
licenseriad under MIT-licensen. Implementationen kan användas med Node.js
eller en webbläsare med en JavaScript-motor som implementerats enligt ECMA-262
version 3.

% vim: set tw=78:ts=2:sw=2:et:fdm=marker:wrap:wm=78:ft=tex
% vim: spell spelllang=sv
