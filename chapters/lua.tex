
\section{Programmeringsspråket Lua}

Lua är ett skriptspråk som kombinerar procedurell programmering med
funktionalitet från det funktionella paradigmet och det objektorienterade
paradigmet \citep{luaimp}.

Målet av Lua är att använda en enkel syntax med en liten grupp av
datastrukturer, att vara snabbt såväl i kompilering som exekvering, att kunna
köras på så många platformer som möjligt och slutligen att vara effektivt för
inbäddningen i programvara. Dessa mål anses vara uppnådda enligt \cite{luaimp}.

Utgående från dessa mål är Lua ett bra alternativ för konfigurationsspråk och
används idag som sådant i bland annat Adobes Photoshop Lightroom och World of
Warcraft \citep{lua}.

\subsection{Syntax och uppbyggnad}

Trots att ett av Luas mål är dess enkelhet har språket en stor mängd
funktionalitet som fås av dess flexibilitet och val av datastruktur.

\subsubsection{Datatyper}

Lua består av 8 datatyper: \textit{nil, boolean, number, string, table,
  function}. Eftersom språket använder sig av dynamisk typning kan variabler
referera till vilken datatyp som helst. Därmed använder språket inte heller
datatypsdefinitioner vid variabel deklarationer \citep[s. 9]{ir06}.

Tabeller är Luas enda datastruktur och detta är egentligen en associativ
array. Detta betyder att den är dynamiskt växande, kan indexeras av vilken
datatyp som helst (förutom \textit{nil}) samt kan även hålla vilket värde som
helst. Med hjälp av denna flexibilitet kan tabeller användas både som en array
och som en hashtabell \citep[s. 15]{ir06}.

\subsubsection{Satser och block}

Lua har stöd för de vanliga satser från exempelvis C såsom if, while, for,
break och return. Ytterligare existerar satser såsom repeat, lokala variabel
deklarationer samt fler-variabel deklarationen. En annan mer unik
funktionalitet för Lua är att funktioner kan returnera flera värden.

En syntaktisk skillnad till språk i likhet med C är hur sats-block definieras.
Istället för att omringa block med klammerparenteser (\{ och \}) använder man
sig av terminaler. Initialiserings terminalen skiljer sig beroende på
sats-typen men alla satser använder \textit{end} som avslutande terminal.

Ett exempel på hur de olika sats-strukturerna ser ut i Lua hittas i
figur~\ref{fig:luacode}.

\begin{figure}[ht]
  \begin{minipage}[t]{0.5\textwidth}
\begin{lstlisting}[language=Lua]
i, value = 11, "part" .. "ial"

if i > 10 then
  value = true
elseif i < 10
  value = false
else
  value = nil
end

for k=0,10,1
  i = i + k
end

function test(o)
  return o, 2
end

i, j = test(1)
\end{lstlisting}
  \end{minipage}
  \begin{minipage}[t]{0.5\textwidth}
\begin{lstlisting}[language=Lua]
while true
  i = i + 1
  if i > 20
    break
  end
end

repeat
  i = i - 1
until i < 10

local days = {"Mon", "Tue", "Wed", "Thu", "Fri", "Sat", "Sun"}
for day in ipairs(days) do
  print(day)
end
\end{lstlisting}
  \end{minipage}
  \caption{Överblick av Luas sats-typer.}
  \label{fig:luacode}
\end{figure}

\subsection{Funktionell programmering}

Ett karaktärsdrag från det funktionella paradigmet är att alla luas datatyper
är värden av första ordningen. Eftersom en funktion är en datatyp kan den såsom
tal och teckensträngar accepteras som funktionsargument samt som returvärde.

När man kombinerar denna funktionalitet med Luas lexikala omfång (scope)
möjliggörs användningen av s.k. \textit{``closures''}. Closures innebär att
en funktion vid exekvering har referenser till variabler vid olika
händelseförlopp. Den har tillgång till dess egna inre variabler vid
exekveringen men också yttrevariabler från den miljö var funktionen
definierats \citep[s. 45]{ir06}.

Användningen av closures är populärt i funktionella programmeringsspråk för
att implementera diverse algoritmer såsom \textit{``curry''}-funktionen.

\subsection{Objektorienterad programmering}

Trots att Lua inte innehåller klasser eller funktionalitet för arv finns det
koncept som möjliggör dess implementation.

\subsubsection{Objekt}

Luas tabeller är objekt och har därmed också funktionalitet för metoder och
för en identitet i betydelsen att ett objekt som ändrar värde alltid är samma
objekt och olikt ett annat objekt med samma värde. Ytterligare kan en metod
definiera referensen till sig själv med hjälp kolon operatorn. Detta gör
tabeller till en nära identisk representation av objekt i ett objektorienterat
språk \citep[s. 149]{ir06}.

\begin{figure}[ht]
  \begin{lstlisting}[language=Lua]
Person = {name = "Charles"}

function Person:greet(name)
  print("Hello " .. name)
  print("My name is " .. self.name)
end
  \end{lstlisting}
  \caption{Ett objekt i Lua}
\end{figure}

\subsubsection{Klasser}

Lua har inget koncept av klasser, alla objekt definierar sin egen struktur och
har inget direkt arv. Vi kan dock simulera klasser och arv genom att använda
en struktur likt prototyp-baserat arv. Detta görs med hjälp av s.k.
metatabeller \citep[s. 151]{ir06}.

Efter följande deklaration kommer värden att sökas i Person-objektet om det
inte hittas hos carl-objektet:
\\

\begin{lstlisting}[language=Lua]
setmetatable(carl, {__index = Person});
\end{lstlisting}
\vspace{-2em}
Vad som återstår är att kombinera funktionaliteten för prototyp-arv samt
objekt metoder till något som liknar klasser. Figur~\ref{fig:class} visar en
klass-lik implementation i Lua.

\begin{figure}[ht]
  \begin{lstlisting}[language=Lua]
function Person:new(name)
  o = { name = name }
  setmetatable(o, self)
  self.__index = self
  return o
end

function Person:greet(name)
  print("Hello " .. name)
  print("My name is " .. self.name)
end

carl = Person.new("Carl")
carl:greet("Roger");
  \end{lstlisting}
  \caption{En arv-implementation i Lua med hjälp av metatabeller.}
  \label{fig:class}
\end{figure}

\subsection{Grammatik}

Luas grammatik existerar i en EBNF notation som inkluderats i dess
ursprungliga form i Bilaga 1.

I version 3.0 övergick Lua från en genererad parser till en handskriven
rekursivt nedstigande parser \citep{luaimp}. På grund av detta gjordes det
förändringar i grammatiken som inte dokumenterats. Uttrycksparsern har
exempelvis omstrukturerats totalt för att kunna beskrivas med en
LL(1)-grammatik \citep[s. 175]{bf09}. Trots förändringen är grammatikens
validering identisk med den ursprungliga versionen. Förändringen gjordes för
att kunna byta ut parsern.

Vid analys av grammatiken kan det konstateras att mycket av grammatiken går att
implementera med en LL(1)-grammatik utan större förändringar. De problem som
uppstår är en vänsterrekursion i uttrycksregeln samt ett lookahead-problem för
att åtskilja variabel deklarationer och funktionskallelser. Dessa problem
kommer hanteras i följande kapitel.

% vim: set tw=78:ts=2:sw=2:et:fdm=marker:wrap:wm=78:ft=tex
% vim: spell spelllang=sv
