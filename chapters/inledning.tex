\section{Inledning}

Det existerar ett flertal olika parser-arkitekturer i dagens kompilatorer.
Vissa använder sig av maskingenererad kod medan andra implementerar
komponenten manuellt. Beroende på komplexiteten av det kompilerade språkets
syntax påverkas även komplexiteten av parser implementationen upp till att
vissa implementationer väljer att inte vara fullständiga.

Rapporten utgår från att läsaren har kunskap inom programmering i allmänhet
samt en förståelse av JavaScript syntax.

\subsection{Målsättning}

Målet med detta examensarbete är att redogöra för hur ett programmeringsspråk
är uppbyggt och hur parsningen av dess syntax implementeras. Varför är vissa
språk svårare att parsa än andra? Vad är orsaken till att vissa kompilatorer
implementerar parser-komponenten manuellt medan andra använder en
maskingenererad komponent? Hur utförs implementationen av en parser i
praktiken och hur komplicerat är det? Detta är frågor som examensarbetet ämnar
besvara.

\subsection{Utförande}

Teoridelen av detta arbete redogör för hur programmeringsspråk är uppbyggt
samt hur de huvudsakliga parsning-arkitekturerna fungerar och används.

Den praktiska delen använder denna teori för att implementera en handskriven
Lua-parser i JavaScript. Orsaken till att JavaScript valts som
implementationsspråk är att i framtiden kunna använda parsern som ett
analyseringsverktyg i nätbaserade kodredigerare.

För att upprätthålla en hög kvalitet på implementationen har diverse
kvalitetssäkringar gjorts upp. Över 500 funktionstest har skapats med hjälp av
kodgenerering och manuell verifiering. Testen baserar sig på Yueliang
projektets testsvit. Med dessa test har ett flertal fel korrigerats och
implementationen har i nuläget en kodtäckning på 100\%.

Funktionstest, mätning av kodtäckning, test för funktionskomplexitet samt en
statisk kodanlys körs innan varje uppdatering för att verifiera dess
kvalitet. Ytterligare används Travis kontinuerlig integrering för att
verifiera att utomstående uppdateringar håller standarden.

Slutprodukten av arbetet finns att hitta på
\url{http://oxyc.github.com/luaparse/} och är licenseriat under MIT-licensen.
Implementationen kan användas med Node.js eller en webbläsare med en
JavaScript-motor som implementerat ECMA-262 version 3.

\subsection{Avgränsning}

Teorin om formella språk innehåller enbart det som krävs för att förstå senare
kapitel. Ytterligare presenteras enbart tekniker som är anses vara aktuella
för parser-implementationen eller Lua språket. Till detta hör inte senare
kompilatorskeden såsom semantisk analys eller kodgenerering.

% vim: set tw=78:ts=2:sw=2:et:fdm=marker:wrap:wm=78:ft=tex
% vim: spell spelllang=sv
