\section{Inledning}

Det existerar ett flertal olika parser-arkitekturer i dagens kompilatorer.
Vissa använder sig av maskingenererad programkod medan andra implementerar
komponenten manuellt. Beroende på komplexiteten av det kompilerade språkets
syntax påverkas även komplexiteten av parser-implementationen av att
vissa implementationer väljer att inte vara fullständiga.

Denna examensarbetsrapport utgår från att läsaren har kunskap inom
programmering i allmänhet samt en förståelse av JavaScript-syntax.

\subsection{Målsättning}

Målet med detta examensarbete är att redogöra för hur ett programmeringsspråk
är uppbyggt och hur parsningen av dess syntax implementeras. Varför är vissa
språk svårare att parsa än andra? Vad är orsaken till att vissa kompilatorer
implementerar parser-komponenten manuellt medan andra använder en
maskingenererad komponent? Hur utförs implementationen av en parser i
praktiken och hur komplicerat är det? Dessa frågor ämnar examensarbetet
besvara.

\subsection{Utförande}

Teoridelen i detta arbete redogör för hur programmeringsspråk är uppbyggt
samt hur de huvudsakliga parsning-arkitekturerna fungerar och används.

Den praktiska delen använder denna teori för att implementera en handskriven
Lua-parser i JavaScript. I kapitel 6 beskrivs en prestandaanalys och parserns
väsentliga flaskhalsar avlägsnas. Orsaken till att JavaScript valts som
implementationsspråk är att i framtiden kunna använda parsern som ett
analyseringsverktyg i en nätbaserad programkodsredigerare.

För att upprätthålla en hög kvalitet på implementationen har ett antal
kvalitetssäkringar uppgjorts. Över 500 funktionstest har skapats med hjälp av
kodgenerering och manuell verifiering. Testen baserar sig på
Yueliang-projektets testsvit. Med dessa test har ett flertal fel korrigerats
och implementationen har i nuläget en programkodstäckning på 100\%.

Funktionstest, mätning av programkodstäckning, test för funktionskomplexitet
samt en statisk programkodsanalys körs före varje uppdatering för att
verifiera dess kvalitet. Ytterligare används Travis kontinuerlig integrering
för att verifiera att utomstående uppdateringar håller standarden.

\subsection{Avgränsning}

Teorin om formella språk innehåller enbart det som krävs för att förstå senare
kapitel. Ytterligare presenteras enbart tekniker som är anses vara aktuella
för parser-implementationen eller Lua-språket. Till detta examensarbete hör
inte senare kompilatorskeden såsom semantisk analys eller
programkodsgenerering.

% vim: set tw=78:ts=2:sw=2:et:fdm=marker:wrap:wm=78:ft=tex
% vim: spell spelllang=sv
