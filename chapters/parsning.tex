\section{Parsning}

Implementeringen av ett programmeringsspråk finns i flera varianter och en
vanlig sådan är kompilatorn. Denna implementation läser input och producerar
sedan ett körbart program utgående från de instruktioner den fått.  Själva
processen av kompilering består av flera faser och komponenter. Den första
komponenten är en parser som läser input och konstruerar en maskinläslig
struktur utgående från den grammatik som givits. Vid detta skede bryr sig
programmet inte ännu om vad som skall göras utan den försöker enbart
identifiera de olika reglerna och granska att dess syntax är korrekt. Parsern
körs normalt som en komponent inne i en tolk vars funktion i sin tur är att
förstå och tolka innebördet hos en regel. När parsern är klar med sin analys
returnerar den strukturen till tolken som i sin tur returnerar sin modifierade
version av strukturen tillbaka till huvudkomponenten, kompilatorn. Kompilatorn
fungerar som en översättare som slutligen genererar den maskinkod som datorns
processor förstår. En översikt av dessa komponenter och dess funktioner
visas i figur~\ref{fig:compiler} \citep[s. 16]{pt10}.

\begin{figure}[ht]
  \includegraphics[width=\textwidth]{figures/output/language-stages.pdf}
  \caption{Översikt av komponenterna i en kompilator.}
  \label{fig:compiler}
\end{figure}

Parsningsprocessen kan delas upp i två skilda faser, först en s.k.
lexikal analys som identifierar lexikala element, som kallas tokens. Tokens
är identifierbara teckensträngar med speciell betydelse. De kategoriseras
enligt typer såsom nyckelord, konstanter, parametrar osv. \citep[s. 6]{aa06}.

Den andra fasen som sker efter identifieringen av tokens är den syntaktiska
analysen där elementen sammansätts till helhetsuttryck granskat enligt
grammatikens produktionsregler.

\subsection{Lexikal analys}

Eftersom elementen i en lexikal analys kan beskrivas med en reguljär grammatik
använder man sig ofta av en ändlig automat för att läsa den. Denna typ av
automat brukar man kalla för lexer. Automaten börjar i ett specifikt startläge
var den väntar på ett tecken att läsa. När ett tecken läses går den genom en
serie alterneringar för att minska mängden slutgiltiga lösningar. När därpå
följande tecken läses in fortsätter den att härleda sig vidare tills den når
en slutgiltig lösning, eller alternativt inte känner igen elementet och skapar
ett felmeddelande. När lösningen är hittad skickar lexern det identifierade
elementet tillbaka till parsern och återgår till sitt utgångsläge för att
vänta på nästa tecken.  På detta sätt kan lexern, som har en effektiv
algoritm, avlägsna onödig information såsom mellanslag och kommentarer för att
sedan ge uttryckets egentliga element vidare till parsern som nu enkelt vet om
en teckensträng är en nyckelordsterminal eller ett tal \citep[s.  51]{sm09}.
Ett lexikaliseringsexempel visas i figur~\ref{fig:lexing}.

\begin{figure}[ht]
  \begin{alltt}
\boxkeyword{if} \boxint{2} \boxpunct{+} \boxint{1} \boxpunct{>} \boxint{3} \boxpunct{==} \boxkeyword{false} \boxkeyword{then}
  -- en kommentar
  \boxkeyword{print} "\boxstring{foo}"\boxpunct{;}
\boxkeyword{end}
  \end{alltt}
  \caption{Ett lexikaliseringsexempel på en if-sats skriven i Lua.
    Blanksteg samt kommentarer ignoreras och tokentyperna är nyckelord, tal,
    specialsymbol och teckensträng.}
  \label{fig:lexing}
\end{figure}

\subsection{Syntaktisk analys}

Processen att parsa input och validera dess syntax enligt en kontextfri
grammatik kallas syntaktisk analys eller enbart parsning. Detta görs
vanligtvis i kombination med en lexikal analys för att förenkla
implementation men kan också genomföras direkt på input \citep[s. 8]{aa06}.

Analysen kombinerar de tokens som identifierats en efter en och försöker hitta
en giltig produktionsregel för kombinationen. Om en produktionsregel
identifierats förväntas alla tokens överensstämma med regeldefinitionens
terminaler och icke-terminaler. Visar det sig att en token inte överensstämmer
skapas ett felmeddelande.

Vid implementationen av en kompilator avbryts parsningen när ett fel
påträffats eftersom syntaxen inte är giltig. Vissa andra implementationer
såsom \textit{``syntax highlighters''} i textredigerare försöker hoppa över
produktionsregeln och fortsätta med nästa eftersom detta ger en bättre
användarupplevelse.

\subsection{Syntaxrepresentation}

Allteftersom produktionsregler parsats i.o.m. en syntaktisk analys skapas en
maskinförståelig representation av dess innehåll. Lättast är det att tänka sig
denna representation som en trädstruktur trots att den inte nödvändigtvis
behöver vara det.

\begin{figure}[ht]
  \includegraphics[width=8cm]{figures/output/tree.pdf}
  \caption{En trädrepresentation av en if-sats skriven i Lua.}
  \label{fig:syntaxtree}
\end{figure}

Varje nod i trädet representeras av en produktionsregel. Löv-noderna är
terminaler (figur~\ref{fig:syntaxtree}). Terminaler såsom \textit{if}, och
\textit{else} utelämnas ofta i en praktisk implementation eftersom de kan
knytas som attribut till en nod \citep[s. 49]{sm09}.

Representationen skapas i den syntaktiska analysatorn när en produktionsregel
identifierats och kan sedan användas i senare skeden såsom i en semantisk
analysator eller i en komponent för programkodsoptimering.

Trädstrukturerna varierar beroende på syftet av parsern. Kompilatorer vill
att de skall vara så nära maskinkod som möjligt medan andra verktyg såsom
statiska programkodsanalysatorer vill att de ska vara på en högre nivå \citep[s.
6]{pt10}.

En vanlig representationsform på en högre nivå är ett abstrakt syntaxträd
(AST).  Detta är verkligen uppbyggt såsom ett träd enligt vad beskrivits
hittills. Karaktärsdrag av ett AST är att det är kompakt, enkelt att förflytta
sig genom och betydelsefullt. Exempelvis kan varje nod existera som ett objekt
i en klass namngiven efter produktionsregeln \citep[s 77]{pt10}.

\subsection{Parsertyper}

Beroende på grammatiken av ett språk krävs det olika algoritmer för att kunna
parsa den. Den s.k. Early-algoritmen kan parsa alla typer av kontextfria
grammatiker i $O(n^3)$ tid, där n är inputlängd \citep[s. 67]{sm09}. De
flesta parsers behöver dock inte en så generell grammatik utan kan parsas i
$O(n)$ tid med hjälp av \textit{``uppifrån-och-ner''}-algoritmer eller
\textit{``nerifrån-och-upp''}-algoritmer. Det existerar ytterligare algoritmer
för olika delmängder av kontextfria grammatiker men de vanligaste bygger på
någon av dessa \citep[s. 61]{aa06}.

När man beskriver en parser nämner man ofta hur långt fram den kan se innan
den gör ett beslut av vilken produktionsregel den skall följa, hur många s.k.
\textit{``lookahead''}-tokens den har. Detta antal skriver man inom en
parentes efter parsertypens namn. Exempelvis skulle den s.k. LL-parsern med 2
\textit{``lookahead''}-tokens skrivas LL(2) \citep[s. 69]{sm09}.

\subsubsection{LL-parser}

LL-parsers härleder regler från vänster och använder sig av
\textit{``uppifrån-och-ner''}-algoritmen. Parsern klarar av en mindre delmängd
av kontextfria grammatiker och man använder benämningen LL-grammatik för att
uttrycka den delmängd som en LL-parser kan parsa.

En LL-parser går att skriva för hand eftersom den i allmänhet följer en logisk
tankegång. Den börjar från en rot-regel och arbetar sig sedan neråt
med hjälp av att jämföra terminaler från vänster, likt en ändlig
automat. När alla alternativa lösningar uteslutits förväntar sig parsern att
löv-reglerna skall passa in en efter en.

\subsubsection{LR-parser}

LR-parsers härleder regler från höger och använder sig av
\textit{``nerifrån-och-upp''}-algoritmen. Denna typ av parser klarar av att
analysera en större delmängd grammatiker än t.ex. LL-parsers och är därmed mer
vanlig i programvara som genererar parsers \citep[s. 61]{aa06}.

LR-parsers genereras huvudsakligen av maskiner eftersom de är svårare att
visualisera. Dessa parsers börjar med att läsa löv-reglerna, alltså de minsta
identifierbara reglerna och ansluter dem sedan till varandra fram till det att
den nått en slutgiltig rot-regel \citep[s. 67]{sm09}.

En steg för steg jämförelse mellan LR- och LL-parserns algoritmer visas i
figur~\ref{fig:ll-vs-lr}.

\begin{figure}[ht]
  \begin{minipage}[t]{0.5\textwidth}
    \Tree [.{\textit{\hspace*{0.5cm}Uttryck}}
  !\qsetw{1.5cm}
]
\\

\Tree [.{\textit{Uttryck}}
  \textbf{2}
  \textbf{+}
  \textit{uttryck}
  !\qsetw{1.5cm}
]
\\

\Tree [.{\textit{Uttryck}}
  \textbf{2}
  \textbf{+}
  [.{\textit{uttryck}}
    \textbf{3}
    \textbf{-}
    \textit{uttryck}
    !\qsetw{1.5cm}
  ]
  !\qsetw{1.5cm}
]
\\

\Tree [.{\textit{Uttryck}}
  \textbf{2}
  \textbf{+}
  [.{\textit{uttryck}}
    \textbf{3}
    \textbf{-}
    [.{\textit{uttryck}}
      \textbf{4}
      !\qsetw{1.5cm}
    ]
    !\qsetw{1.5cm}
  ]
  !\qsetw{1.5cm}
]

  \end{minipage}%
  \begin{minipage}[t]{0.5\textwidth}
    \textbf{2}

\textbf{2} \textbf{+}

\textbf{2} \textbf{+} \textbf{3}

\textbf{2} \textbf{+} \textbf{3} \textbf{-}

\textbf{2} \textbf{+} \textbf{3} \textbf{-} \textbf{4}
\\

\textbf{2} \textbf{+} \textbf{3} \textbf{-} \Tree [.{\textit{uttryck}}
  \textbf{4}
]
\\

\textbf{2} \textbf{+} \Tree [.{\textit{uttryck}}
  \textbf{3}
  \textbf{-}
  [.{\textit{uttryck}}
    \textbf{4}
  ]
]
\\

\Tree [.{\textit{Uttryck}}
  \textbf{2}
  \textbf{+}
  [.{\textit{uttryck}}
    \textbf{3}
    \textbf{-}
    [.{\textit{uttryck}}
      \textbf{4}
    ]
  ]
]

  \end{minipage}%
  \caption{Steg för steg parsning av ett matematiskt uttryck med en
    ``uppifrån-och-ner''-algoritm (vänster) och en
    ``nerifrån-och-upp''-algoritm (höger).}
  \label{fig:ll-vs-lr}
\end{figure}

\subsubsection{LALR-parser}

LALR-parsern baserar sig på en LR(0)-parsers struktur men med stöd för
\textit{``lookahead''}-tokens. Skillnaden mellan LR-parsers med stöd för
\textit{``lookahead''}-tokens och LALR-parsern är antalet tillstånd
\textit{``push-down''}-automaten har.  Eftersom LALR-parsern kräver färre
tillstånd och dessutom kräver mindre minne kan den ses som en förenklad och
mer effektiv version \citep[s. 266]{aa06}.

\subsubsection{Rekursivt nedstigande parser}

En annan typ av parser som använder sig av
\textit{``uppifrån-och-ner''}-algoritmen är den rekursivt nedstigande parsern.
Denna parsertyp klarar av att parsa LL-grammatik och är därför vanlig för
handskrivna parsers. Istället för att utnyttja en automat fungerar den genom
att knyta varje icke-terminal i grammatiken till en egen funktion som ansvarar
för att identifiera dess löv i grammatiken. För varje icke-terminal i sin egen
regel anropar den rekursivt vidare dess funktioner fram till det att rot
funktionen får hela syntaxträdet returnerat \citep[s. 24]{pt10}.

Flödet av en rekursivt nedstigande parser visas i
figur~\ref{fig:recursivedescent} där varje nod symboliserar en funktion.
En röd pil symboliserar dess funktionsanrop medan en blå pil symboliserar
funktionens returnering.

\begin{figure}[ht]
  \includegraphics[width=12cm]{figures/output/recursivedescent.pdf}
  \caption{Flöde av en rekursivt nedstigande parser steg för steg.}
  \label{fig:recursivedescent}
\end{figure}

\subsection{Eliminering av mångtydighet}

Eftersom handskrivna parsers i huvudsak använder sig av
\textit{``uppifrån-och-ner''} algoritmer kan de inte parsa alla kontextfria
grammatiker. Vissa grammatiker innehåller mångtydigheter som måste elimineras
för att denna typ av algoritm skall kunna användas.

\subsubsection{Vänsterrekursion}

Ett vanligt problem med grammatiken i en parser med \textit{``uppifrån-och-ner''}
algoritmen är vänsterrekursion. Detta innebär att produktionsregel använder
sig själv som icke-terminal längst till vänster i sammansättningen. När
detta inträffar kommer implementationen att fastna i en oändlig upprepning.
För att implementera en LL-parser måste man därför eliminera vänsterrekursion.

Exempelvis är produktionsregeln för kalkylatorn i figur~\ref{fig:cfg} på sida
\pageref{fig:cfg} vänsterrekursiv.

\setlength{\grammarindent}{5em}
\begin{grammar}
  \singlespace\small%
  \fontfamily{lmr}\selectfont
  <uttryck> ::= <tal>
    \alt "(" <uttryck> ")"
    \alt <uttryck> <operator> <uttryck>
\end{grammar}

Genom att expandera rekursionen i alla existerande villkor som nya villkor får
vi:

\setlength{\grammarindent}{5em}
\begin{grammar}
  \singlespace\small%
  \fontfamily{lmr}\selectfont
  <uttryck> ::= <tal>
    \alt "(" <uttryck> ")"
    \alt <tal> <operator> <uttryck>
    \alt "(" <uttryck> ")" <operator> <uttryck>
\end{grammar}

Detta beskriver fortfarande en identisk grammatik och följande steg är att
förenkla uttrycket med en ny produktionsregel som utnyttjar $\epsilon$, en
symbol för att representera ett tomt värde.

\setlength{\grammarindent}{5em}
\begin{grammar}
  \singlespace\small%
  \fontfamily{lmr}\selectfont
  <uttryck> ::= <tal> <uttryck'>
    \alt "(" <uttryck> ")" <uttryck'>

  <uttryck'> ::= <tal>
    \alt "(" <uttryck> ")"
    \alt <operator> <uttryck>
    \alt $\epsilon$
\end{grammar}

Grammatikens funktion är fortfarande identisk men vänsterrekursionen som är
omöjlig att utföra i en \textit{``uppifrån-och-ner''} algoritm är eliminerad.

En förenklad matematisk regel som kan användas för att eliminera
vänsterrekursionen $A \rightarrow A\alpha\;|\;\beta$ är \citep[s. 212]{aa06}:
\begin{align*}
A &\rightarrow \beta A^\prime \\
A^\prime &\rightarrow \alpha A^\prime\;|\;\epsilon
\end{align*}

\subsubsection{Vänsterfaktorering}

Ytterligare ett problem \textit{``uppifrån-och-ner''}-algoritmen är
produktionsregler där två villkor påminner om varandra i början av
definitionen.

Exempelvis behövs en obestämd mängd \textit{``lookahead''}-tokens för att
identifiera det korrekta villkoret i följande produktionsregel:

\setlength{\grammarindent}{5em}
\begin{grammar}
  \singlespace\small%
  \fontfamily{lmr}\selectfont
  <if> ::= "if" <uttryck> "then" <block> "else" <block> "end"
    \alt "if" <uttryck> "then" <block> "end"
\end{grammar}

Med hjälp av vänsterfaktorering kan vi omvandla regeln till:

\setlength{\grammarindent}{5em}
\begin{grammar}
  \singlespace\small%
  \fontfamily{lmr}\selectfont
  <if> ::= "if" <uttryck> "then" <block> <else>

  <else> ::= "else" <block> "end"
    \alt "end"
\end{grammar}


Den matemtiska regeln för liknande uttryck i formen $A \rightarrow
\alpha\beta_1\;|\;\alpha\beta_2 $ är \citep[s. 214]{aa06}:
\begin{align*}
A &\rightarrow \alpha A^\prime \\
A^\prime &\rightarrow \beta_1\;|\;\beta_2
\end{align*}

\subsection{Vidareutveckling av parsertyper}

Genom att eliminera mångtydigheter i en grammatik klarar
\textit{``uppifrån-och-ner''}-baserade parsers av en större mängd
programmeringsspråk. I vissa fall kan mångtydigheter inte elimineras, vilket
leder till att parsern behöver ett större antal \textit{``lookahead''}-tokens
för att kunna göra ett beslut.

\subsubsection{Backtracking}

I vissa fall är det användbart att inte ha en begränsad mängd
\textit{``lookahead''}-tokens och då kan man använda sig av en metod som heter
\textit{``backtracking''} förutsatt det handlar om en rekursivt nedstigande
parser.

Implementationen av en \textit{``backtracker''} görs genom att markera
positionen var en mångtydighet uppstår och sedan välja ett regelalternativ.
När alternativet bevisats vara korrekt eller inkorrekt återgår parsern till
den markerade positionen. Om ett felmeddelande skapats försöker parsern med
följande alternativ och repeterar detta fram till att en regel visar sig vara
korrekt varefter den påbörjar den egentliga parsningen utan att markera
positionen \citep[s. 57]{pt10}.

För att uttrycka att en parser kan se en obegränsad mängd tokens framåt anger
man antalet som k i LL(k).

\subsubsection{Packrat parsning}

En nackdel med den tidigare nämnda \textit{``backtracking''}-metoden är att
parsern alltid måste återgå till det tidigare tillståndet och därefter påbörja
en andra parsning av regeln. Detta ökar den tidsmängd som krävs för att parsa
ett inputvärde.

Packrat parsning är en annan typ av parser vars metod går att implementera för
en rekursivt nedstigande parser. Den fungerar likt
\textit{``backtracking''}-metoden men skiljer sig genom att memorera
resultatet av parsningsträdet samt positionen var parsningen avslutades. När
parsern återgår till det tidigare tillståndet granskar den om parsningen var
framgångsrik. Om den visar sig vara, använder parsern sig av det redan
memorerade parsningsträdet och hoppar vidare till positionen där parsningen
avslutades.

\subsection{Parser-generatorer}

Att implementera en parser för hand är en tidskrävande och felbenägen process.
Produktionsregler påminner mycket om varandra och repetitionen att
implementera varje regel blir en långdragen uppgift med möjlighet för
introduktion av fel vid varje steg. På grund av detta har det blivit populärt
att använda en s.k. parser-generator som förser programmeraren med en färdig
parser-implementation utgående från en given grammatik \citep[s. 26]{pt10}.

Med hjälp av en parser-generator kan implementatören upprätthålla grammatiken
av ett språk genom att skriva och modifiera en BNF-liknande syntax istället
för den egentliga programkoden. Detta har stora fördelar om grammatiken inte är helt
stabil eftersom enbart några rader behöver korrigeras vid förändringar. För
ett språk med en stabil grammatik existerar fördelarna endast vid
implementationsskedet och kan vara en nackdel i situationer som felsökning och
skapandet av läsbara felmeddelanden för syntaxfel \citep[s. 175]{bf09}.

Uppdelningen av genererade parsers och handskrivna parsers är delad, de flesta
språk använder sig av en genererad parser men några såsom C-språkets GCC \citep{gcc}
samt Luas egna parser \citep{luaimp} har börjat från en genererad parser och
sedan övergått till en handskriven rekursivt nedstigande parser för utökad
funktionalitet.

Parser-generatorer existerar för flera språk samt för diverse grammatik- och
algoritmtyper. I detta kapitel kommer jag att presentera två
parser-generatorer som är vanliga i unix-system.

\subsubsection{Lex}

Lex är en generator för reguljära grammatiker skapad av Mike Lesk och
Eric Schmidt. Verktyget använder en egen grammatiksyntax för att
definiera produktionsregler i ett språk och generera sedan en lexer i C eller
Ratfor. För varje produktionsregel kan användaren definiera en åtgärd som
kommer att exekveras när en regel identifierats. Vanligen är detta C-kod för
att returnera värdet till en syntaktisk analysator. Eftersom Lex genererar
programkod existerar funktionalitet för att inkludera valfri C-kod som
kopieras till resultatfilen. Detta kan användas till att inkludera bibliotek
eller definiera funktioner som produktionsreglerna använder \citep{lex}.

I figur~\ref{fig:lex} visas en enkel grammatikspecifikation som läser input
och identifierar variabeldeklarationer av strängar eller heltal samt ignorerar
blanksteg.

\begin{figure}[ht]
  \lstinputlisting[title="",keywords={printf,return,include,main}]%
    {figures/tex/lex.l}
  \caption{En Lex-specifikation för att identifiera variabeldeklarationer av
    strängar eller heltal.}
  \label{fig:lex}
\end{figure}

En fritt tillgänglig version av Lex är Flex som distribueras av bl.a.
GNU-projektet. I allmänhet är Flex-kompatibel med Lex och är därför ett
populärt alternativ \citep[s. 279]{bd92}.

\subsubsection{Yacc}

Eftersom de flesta programmeringsspråk använder sig av en kontextfri grammatik
är inte Lex verktyget tillräkligt utan man behöver ytterligare en syntaktisk
analysator. Yacc är en parser-generator för kontextfri grammatik skapad av
Stephen Johnson. Syntaxen för grammatikspecifikationen liknar mycket Lex.
Användaren definierar produktionsregler samt deras åtgärder men har nu
möjligheten att använda rekursion. Ytterligare bör man definiera en
lexikaliseringsfunktion, exempelvis \textit{yylex()}, samt vilka tokentyper
som existerar. Integreringen måste ske både i den valfria lexern och i
Yacc-specifikationen, t.ex. med en delad definitionsfil av tokens som lexern
returnerar och den Yacc-genererade parsern kan identifiera. Ett
pseudokod-exempel på ett språk som tillåter variabeldeklarationer av heltal
och strängar visas i figur~\ref{fig:yacc}. För att parsern skall fungera
måste logiken för \textit{input} samt lexikaliseringen av \textit{VARIABLE},
\textit{STRING}, \textit{INTEGER} och \textit{EQUAL_SIGN} definieras i lexer
koden.

GNU-projektet har även skapat en \textit{``copyleft''}-licenseriad version av Yacc som
heter GNU bison. Bison är i allmänhet kompatibel med Yacc-syntax och är därför
ett populärt alternativ \citep[s. 277]{bd92}.

\begin{figure}[ht]
  \lstinputlisting[title="",keywords={printf,return,include,main,do,while}]%
    {figures/tex/yacc.y}
  \caption{En Yacc-specifikation för variabeldeklarationer av heltal och
    strängar.}
  \label{fig:yacc}
\end{figure}

\subsubsection{Jison}

Jison är en JavaScript-programmerad parser-generator för både reguljära grammatiker
och kontextfria grammatiker. Verktyget använder samma koncept som Lex/Flex
samt Yacc/Bison och därmed är grammatik-specifikationen samma. Skillnaden är
att specifikationen kan innehålla grammatik för både lexern och den
syntaktiska analysatorn. Trots att det finns funktionalitet för att generera
både en lexer och en syntaktisk analysator tillåter Jison att användaren
utnyttjar en separat lexer som följer ett specificerat gränssnitt
\citep{jison}.

% vim: set tw=78:ts=2:sw=2:et:fdm=marker:wrap:wm=78:ft=tex
% vim: spell spelllang=sv
