\section{Språkteori}

För att en dator skall ha möjlighet att förstå innebörden i ett uttryck krävs
det att uttrycket är uppbyggt med en konsekvent utformning av såväl dess
syntax och dess semantik. Detta krav existerar inte i naturliga språk som
svenska utan existerar för en specifik typ av språk, de formella språken.
Formella språk är uppbyggda enligt matematiska regler som definierar språkets
alfabet och hur alfabetsymboler kan kombineras för att skapa uttryck. Teorin
härstammar från språkvetenskap men har idag en stor betydelse inom
datavetenskap eftersom den utnyttjas för att konstruera programmeringsspråk
\citep[s. 41]{sm09}.

\subsection{Backus-Naur Form}

Inom datavetenskap är syntax den kombination av tecken som är giltig för att
skapa ett uttryck. Uttryckets funktion kan variera, t.ex. kan funktionen
vara en del av ett flöde som skapar ett datorprogram, eller enbart ett format
för att uttrycka konfigurationer. Processen att läsa denna syntax och granska
om den är giltig kallas syntaktisk analys eller parsning.

När man beskriver syntaxen av ett språk använder man sig av ett metaspråk för
att definiera de syntaktiska regler man är tillåten att använda. Det finns ett
flertal metaspråk men ett av de vanligaste inom programmeringsspråk är
Backus-Naur Form (BNF) \citep[s. 27]{gd08}. Notationen är uppbyggd enligt
produktionsregler som var för sig definierar en tillåten sammansättning av
teckensträngar som kallas terminaler eller icke-terminaler. Terminaler är
teckensträngar som inte refererar vidare till andra teckensträngar medan
icke-terminaler är sekvenser av terminaler som bildar giltiga
produktionsregler, eller språkliga meningar. Ytterligare existerar vissa
symboler för att uttrycka vilken typ av sammansättningsfunktion är tillåten.
% @TODO liknelsen med meningar är något oklart.

BNF i sig existerar dessutom i flera varianter där vissa varianter är strikta och
ämnade att läsas av maskiner, medan andra varianter försöker visualisera elementen
för en mänsklig läsare. Extended BNF (EBNF) som är en utökad variant av den
ursprungliga BNF-notationen skriver icke-terminaler inom vinkelparentes i
formen ${\langle}regel{\rangle}$. En produktionregels namn, som är en
icke-terminal,
skrivs längst till vänster följt av symbolerna $::=$ samt själva regeln.
Terminalerna skrivs med fet stil och sammansättningarna skrivs med normal
stil samt regelns specifika syntax. De vanliga funktionerna är alternering, som
skrivs med ett lodrätt streck ($|$) mellan alternativen, repetition som skrivs
med en klammerparentes ($\{ \ldots \}$) omkring uttrycket och slutligen
valfrihet som skrivs med en hakparentes ($[ \ldots ]$) runt uttrycket
\citep[s. 28]{gd08}. Dessa är de viktigaste elementen i BNF men det existerar
även övriga funktioner för bl.a. bekvämlighet och läsbarhet.

\subsection{Chomskyhierarkin}

När man skriver grammatiken till ett språk beaktar man alltid vilka
typer av regler man vill tillåta i specifikationen. Utgående från valet man
gör kommer grammatiken och därmed också språket tillhöra en av fyra språkliga
delmängder som sträcker sig från enkel till komplicerad \citep[s. 19]{gd08}.

\begin{figure}[ht]
  \includegraphics[width=5cm]{figures/output/chomsky.pdf}
  \caption{De fyra nivåerna i Chomskyhierarkin}
\end{figure}

Denna indelning kallas för Chomskyhierarkin och används bl.a. för att ta reda
på vilken typ av automat som krävs för att läsa språket. Typen av automat
påverkar komplexiteten av implementationen samt tidsmängden som krävs för att
läsa språket.

Delmängderna börjar från den enklaste typen, reguljär grammatik. Den
reguljära grammatiken är sedan en delmängd av den kontextfria grammatiken som i
sin tur är delmängd till den kontextkänsliga grammatiken. Slutligen tillhör
alla de tidigare nämnda även den obegränsade grammatiken som kan
beskriva alla grammatiker vilka accepteras av en Turingmaskin.

De två huvudsakliga grupperna i dagens programmeringsspråk är dock de två
innersta, reguljära grammatiker samt kontextfria grammatiker \citep[s.
100]{sm09}. Dessa är möjliga att skriva både för hand och av maskiner och har
därför blivit mycket vanliga i design av programmeringsspråk. Majoriteten av
språk använder sig av en kontextfri syntax. Ett undantag är C++ vars
uttryck inte kan definieras enbart utgående från syntaxen utan kräver också en
semantisk analys av ett flertal uttryck. På grund av detta är C++ grammatiken
kontextkänslig och därmed också svår att parsa. I flera fall har
parserimplementationer valt att ignorera mångtydigheterna på grund av deras
komplexitet och deras obetydliga användning \citep[s. 2]{rt05}.

\subsubsection{Reguljär grammatik}

Den innersta delmängden i Chomskyhierarkin är reguljär grammatik och kan uttryckas
enbart m.h.a. reglerna sammanfogning, alternering och repetition. I
programmeringsspråk används ofta en reguljär grammatik för att identifiera
s.k. lexikala element och går att läsa med en ändlig automat \citep[s.
100]{sm09}.

% @TODO polaritet av tal?
För att beskriva alla variationer av ett tal i en kalkylator kan man
använda sig av EBNF grammatiken i figur~\ref{fig:reg}. Ett tal definieras som
alterneringen av ett heltal och ett reellt tal samt dess polaritet. Ett heltal
måste bestå av minst en siffra medan ett reellt tal kan bestå av antingen ett
heltal samt en exponent eller ett decimaltal och en valfri exponent. Detta
innebär att uttrycken \textit{0.14E-2} och \textit{3} är giltiga medan
uttrycket \textit{222e} inte är giltigt eftersom en exponent måste avslutas
med ett heltal. Dessutom måste man tänka på att ett giltigt decimaltal i en
riktig kalkylator inte nödvändigtvis behöver börja med en siffra utan kan
börja med en punkt, dock måste det antingen börja med en siffra eller avslutas
med en siffra eftersom ett uttryck enbart innehållande en punkt inte kan
räknas som giltigt. Alla dessa regler kan bli komplicerade att hålla reda på
och därför underlättar det att arbeta med BNF notationer för att inte mista
giltiga uttryck.

\begin{figure}[ht]
  \begin{grammar}
    \singlespace\small%
    \fontfamily{lmr}\selectfont

    <tal> ::= ["+" | "-"] (<heltal> | <reellt tal>)

    <heltal> ::= <siffra> \{<siffra>\}

    <reellt tal> ::= <heltal> <exponent>
      \alt <decimaltal> [<exponent>]

    <decimaltal> ::= <heltal>.<heltal>

    <exponent> ::= ("e" | "E") ["+" | "-"] <heltal>

    <siffra> ::= "0" | "1" | "2" | "3" | "4" | "5" | "6" | "7" | "8" | "9"

  \end{grammar}
  \caption{EBNF grammatik för ett tal.}
  \label{fig:reg}
\end{figure}

\subsubsection{Kontextfri grammatik}

Tillåter man ytterligare rekursion i en giltig regel är grammatiken inte
längre reguljär, utan klassas som en kontextfri grammatik och måste läsas av
en s.k. \textit{``push-down''} automat ofta kallad parser. Skillnaden från en
ändlig automat är att \textit{``push-down''} automaten har en stack för att
hjälpa i valet av tillstånd \citep[s.  100]{sm09}. Rekursion innebär att en
produktionsregel kan innehålla sig själv som en icke-terminal i
regeldefinitionen. Denna funktionalitet är användbar när ett uttryck skall
vara flexibelt, exempelvis i en kalkylator var ett uttryck kan bestå av ett
tal, en matematisk operation samt en oändlig uppsättning av dessa.

I figur~\ref{fig:cfg} har vi ett exempel på en kalkylator som kan uttrycka
alla dessa funktionaliteter genom att rekursivt hänvisa till sig själv och
därmed tillåta uttryck så som \mbox{\textit{1 + (2 / 7) * -3}}.

\begin{figure}[ht]
  \begin{grammar}
    \singlespace\small%
    \fontfamily{lmr}\selectfont

    <uttryck> ::= <tal>
      \alt "(" <uttryck> ")"
      \alt <uttryck> <operator> <uttryck>

    <operator> ::= + | - | * | /

  \end{grammar}
  \caption{EBNF grammatik för en kalkylator.}
  \label{fig:cfg}
\end{figure}

% vim: set tw=78:ts=2:sw=2:et:fdm=marker:wrap:wm=78:ft=tex
% vim: spell spelllang=sv
