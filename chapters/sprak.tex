\section{Språk teori}

För att en dator skall ha möjlighet att förstå innebördet i ett uttryck krävs
det att uttrycket är uppbyggt med en konsekvent utformning av såväl dess
syntax och dess semantik. Detta krav existerar inte i de naturliga språk så
som svenska utan är unikt för en specifik typ av språk, de formella språken.
Formella språk är uppbyggda enligt matematiska regler som definierar sitt
alfabet och hur det kan kombineras för att skapa uttryck. Teorin härstammar
från språkvetenskap men har idag en stor betydelse inom datavetenskap eftersom
det utnyttjas för att konstruera programmeringsspråk \citep[s. 41]{sm09}.

\subsection{Backus-Naur Form}

Inom datavetenskap är syntax den kombination av tecken som är giltig för att
skapa ett uttryck. Uttryckets funktion kan vara av flera former, t.ex. kan det
vara en del av ett flöde som skapar ett datorprogram, eller enbart format för
att uttrycka konfigurationer. Processen av att läsa denna syntax och granska
om den är giltig kallas för syntaktisk analys eller parsning.

När man beskriver syntaxen av ett språk använder man sig av ett metaspråk för
att definiera de syntaktiska regler man är tillåten att använda. Det finns ett
flertal metaspråk men ett av de vanligaste inom programmeringsspråk är
Backus-Naur Form (BNF) \citep[s. 27]{gd08}. Notationen är
uppbyggd enligt produktionsregler som var för sig definierar en sammansättning
av terminaler och icke-terminaler tillåtna för sin regel. Terminaler är ett
begränsat alternativ teckensträngar, som kan liknas till ord i ett språk.
Icke-terminaler är serier av terminaler som utgör en giltig produktionsregel,
eller en språklig mening. Ytterligare existerar vissa symboler för att
uttrycka vilken typ av sammansättningsfunktion som är tillåten.

BNF i sig existerar dessutom i flera varianter var vissa är mer strikta och ämnade till att
läsas av maskiner, medan vissa försöker visualisera elementen för en mänsklig
läsare. Extended BNF (EBNF) som är en utökad variant av den ursprungliga BNF
notationen skriver icke-terminaler inom vinkelparentes i formen
${\langle}regel{\rangle}$. En produktionsregels namn, som är en icke-terminal,
skrivs längst till vänster följt av symbolerna $::=$ samt själva regeln.
Terminalerna skrivs med fet stil och sammansättningarna skrivs med normal
stil samt regelns specifika syntax. De vanliga funtionerna är alternering, som
skrivs med ett lodrätt streck ($|$) mellan alternativen, repetition som skrivs
med en klammerparentes ($\{ \ldots \}$) omkring uttrycket och slutligen
valfrihet som skrivs med en hakparentes ($[ \ldots ]$) runt uttrycket
\citep[s. 28]{gd08}. Dessa är de viktigaste elementen i BNF men det existerar
även övriga funktioner för bl.a. bekvämlighet och läsbarhet.

\subsection{Chomskyhierarkin}

När man skriver grammatiken till ett språk beaktar man alltid vilka
typer av regler man vill tillåta i specifikationen. Utgående från valet man
gör kommer grammatiken och därmed också språket tillhöra en av fyra språkliga
delmängder som sträcker sig från simpel till komplicerad \citep[s. 19]{gd08}.

\begin{figure}[ht]
  \includegraphics[width=5cm]{figures/output/chomsky.pdf}
  \caption{De fyra nivåerna i Chomskyhierarkin}
\end{figure}

Denna indelning kallas för Chomskyhierarkin och används bl.a. för att ta reda
på vilken typ av automat som krävs för att läsa språket.

Delmängderna börjar från den simplaste typen, reguljär grammatik. Den
reguljära grammatiken är sedan en delmängd av den kontextfria grammatiken som i
sin tur är delmängd till den kontextkänsliga grammatiken. Slutligen tillhör
alla de tidigare nämnda även den obegränsade grammatiken som kan
beskriva alla grammatiker vilka accepteras av en Turingmaskin.

De två huvudsakliga grupperna i dagens programmeringsspråk är dock de två
minsta, reguljära grammatiker samt kontextfria grammatiker \citep[s.
100]{sm09}. Dessa är möjliga att skriva både för hand och av maskiner och har
därför blivit mycket vanliga i design av programmeringsspråk. Majoriteten av
språk använder sig av en kontextfri syntax grammatik. Ett undantag är C++ vars
uttryck inte kan definieras enbart utgående från syntaxen utan kräver också en
semantisk analys av ett flertal uttryck. På grund av detta är C++ grammatiken
kontextkänslig och därmed också svår att parsa. I flera fall har parser
implementationer valt att ignorera mångtydigheterna på grund av dess
komplexitet och dess lilla användning \citep[s. 2]{rt05}.

\subsubsection{Reguljär grammatik}

Den minsta delmängden i Chomskyhierarkin är reguljär grammatik och kan uttryckas
enbart mha. reglerna \textit{sammanfogning}, \textit{alternering} och
\textit{repetition}. I programmeringsspråk används ofta en reguljär grammatik
för att identifiera tokens och går att läsa med en ändlig automat \citep[s.
100]{sm09}.

För att beskriva alla variationer av ett nummer i en kalkylator kan man
använda sig av EBNF grammatiken i figur~\ref{fig:reg}. Ett nummer definieras
som alterneringen av ett heltal och ett reellt tal. Ett
heltal måste bestå av minst en siffra medan ett reellt tal kan bestå
av antingen ett heltal samt en exponent eller ett decimaltal och en valfri
exponent. Detta innebär att uttrycken \textit{0.14E-2} och \textit{3} är
giltiga medan uttrycket \textit{222e} inte är giltigt eftersom en exponent
måste avslutas med ett heltal. Dessutom måste man tänka på att ett giltigt
decimaltal inte nödvändigtvis behöver börja med en siffra utan kan börja med
en punkt, dock måste det antingen börja med en siffra eller avslutas med en
siffra eftersom ett uttryck enbart innehållande en punkt inte kan räknas som
giltigt. Alla dessa regler kan bli komplicerade att hålla reda på och därför
underlättar det att arbeta med BNF notationer för att inte mista giltiga
uttryck.

\begin{figure}[ht]
  \begin{grammar}
    \singlespace\small%
    \fontfamily{lmr}\selectfont

    <nummer> ::= <heltal> | <reellt tal>

    <heltal> ::= <siffra> \{<siffra>\}

    <reellt tal> ::= <heltal> <exponent> | <decimaltal> [<exponent>]

    <decimaltal> ::= { <siffra> } ( "." <siffra> | <siffra> "." ) { <siffra> }

    <exponent> ::= ("e" | "E") ["+" | "-"] <heltal>

    <siffra> ::= "0" | "1" | "2" | "3" | "4" | "5" | "6" | "7" | "8" | "9"

  \end{grammar}
  \caption{EBNF grammatik för ett nummer.}
  \label{fig:reg}
\end{figure}

\subsubsection{Kontextfri grammatik}

Tillåter man ytterligare \textit{rekursion} som en giltig regel är grammatiken
inte längre reguljär, utan klassas som en kontextfri grammatik och måste
läsas av parser \citep[s. 100]{sm09}. Rekursion innebär att en produktionsregel kan innehålla sig
själv som en icke-terminal i regel definitionen. Denna funktionalitet är
användbar när ett uttryck skall vara flexibelt, exempelvis i en aritmetisk
kalkylator var ett uttryck kan bestå av en siffra, en matematisk operation
samt en oändlig uppsättning av dessa.

I figur~\ref{fig:cfg} har vi ett exempel på en kalkylator som kan uttrycka
alla dessa funktionaliteter genom att rekursivt hänvisa till sig själv och
därmed tillåta uttryck så som \textit{1 + (2 / 7) * -3}.

\begin{figure}[ht]
  \begin{grammar}
    \singlespace\small%
    \fontfamily{lmr}\selectfont

    <uttryck> ::= <nummer> | ("-" | "+") <uttryck> | "(" <uttryck> ")"
      \alt <uttryck> <operator> <uttryck>

    <operator> ::= + | - | * | /

  \end{grammar}
  \caption{BNF grammatik för en kalkylator.}
  \label{fig:cfg}
\end{figure}

% vim: set tw=78:ts=2:sw=2:et:fdm=marker:wrap:wm=78:ft=tex
% vim: spell spelllang=sv
