\section{Resultat}

För att mäta resultat av examensarbetets parser-implementation kommer jag att
göra en prestandaanalys av alternativa parsers.

\subsection{Jison-genererad parser}

Parser-generatorn Jison kommer att användas för att generera en Lua-parser som
utnyttjar en \textit{``nerifrån-och-upp''}-algoritm. Grammatiken som används
är baserad på Lua.js-projektets Lua-grammatik. Eftersom Lua.js kompilerar
Lua-kod till JavaScript-kod har grammatiken förenklats till att enbart parsa
och inte lagra någon information från processen. Den slutliga grammatiken
visas i bilaga 3.

I det tidigare kapitlet framgick det att lexern kan utgöra 40\% av en
parsningsprocess. Eftersom den Jison-genererade lexern använder sig av
reguljära uttryck kan den förväntas vara ineffektiv. I ett försök att
förbättra resultatet kommer prestandaanalysen av den genererade parsern
existera i två versioner. En version kommer att använda den fullständiga
parsern medan en annan kommer att utnyttja lexern som utvecklats i detta
examensarbete.

\subsection{LuaMinify}

LuaMinify är en källkods-komprimerare för Lua implementerad i Lua. Projektet
är gjort som ett hobbyprojekt och använder sig en av handskriven rekursivt
nedstigande parser. Implementationen av parsern har inte prioriterat
optimering, vilket är förståeligt eftersom en källkods-komprimerare inte har
ett behov av att vara snabb. För att få en mer komplett bild av prestandan för
handskrivna parsers inkluderas denna parser i analysen.

Prestandaanalysen av parsern från LuaMinify kommer att göras i två versioner.
En version kommer att köras med Luas standardkompilator och en annan med Luas
JIT-kompilator.

\subsection{Mätning}

JavaScript-mätningar görs i Node.js v0.8.13 med Benchmark.js-verktyget och
skriptet för mätningen visas i bilaga 4. Lua-mätningen är implementerad i Lua
och använder sig av socket-biblioteket för att få tillgång en högre
noggrannhet. Skriptet för mätningen visas i bilaga 5. Alla parsningar görs på
ParseLua.lua från LuaMinify-projektet.

\subsection{Slutsats}

\begin{figure}[ht]
  \includegraphics[width=15cm]{figures/output/parsers.pdf}
  \caption{Prestandaresultat från parsning av ParseLua.lua.}
  \label{fig:parserresult}
\end{figure}

Resultatet från tabell~\ref{tab:parsers} visualiseras i
figur~\ref{fig:parserresult} och visar att examensarbetets
parser-implementation har den snabbaste implementation både när det kommer
till lexikal analys och syntaktisk analys.

\begin{table}[ht]
  \caption{Resultat från prestandaanalysen mellan parsers.}
  \begin{tabular}{l l l}
    Parser & Tid (ms) & Standardavvikelse \\
    \hline
    Examensarbetets implementation & 5.6 & 0.5 \\
    Fullständig Jison-genererad implementation & 111.0 & 7.5 \\
    Jison-genererad implementation utan lexer & 49.1 & 2.2 \\
    LuaMinify med Lua 5.1 & 184.1 & 10.7 \\
    LuaMinify med Luajit 2.0.0 & 75.3 & 7.9
  \end{tabular}
  \label{tab:parsers}
\end{table}


Parsern som genererats av Jison visar sig vara 20 gånger långsammare än
examensarbetets implementation.  Genom att byta ut lexern med examensarbetets
lexer halveras exekveringstiden för testet. Lexer är den mest
prestandaoptimerade komponenten i implementationen medan den syntaktiska
analysatorn är en typisk rekursivt nedstigande parser. Eftersom den
genererade parsern fortfarande är långsam jämfört med examensarbetets
implementation kan det konstateras att lösningen inte är passande för ett
prestandakritiskt projekt.

Parsern tillhörande LuaMinify visar sig vara över 30 gånger långsammare än
examensarbetets parser. Detta resultat är svårt att tolka eftersom
implementationerna använder sig av skilda språk. LuaJIT är känd för att vara
en mycket snabb tolk jämfört med andra. I en diskussion från 2010 hävdar dess
skapare Mike Pall \citep{mp10} att skräpinsamling är ett problem för LuaJIT
jämfört med V8. Eftersom parsningsprocessen konstruerar 5000 noder i ett
syntaxträd vid varje test och testen körs 100 gånger är det möjligt att detta
är en orsak till det sämre resultatet. Skräpinsamlingsproblem går vanligtvis
att optimera genom att inte kontrollera användningen av de objekt som skapas
under en exekvering.

% vim: set tw=78:ts=2:sw=2:et:fdm=marker:wrap:wm=78:ft=tex
% vim: spell spelllang=sv
